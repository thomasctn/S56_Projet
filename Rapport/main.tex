\documentclass[a4paper, 11pt]{report}
\usepackage[utf8]{inputenc}   % Pour lire les fichiers UTF-8
\usepackage[T1]{fontenc}      % Pour une meilleure gestion des caractères accentués
\usepackage{textcomp}         % Pour quelques symboles supplémentaires
\usepackage{lmodern}          % Police compatible avec T1
\usepackage[english,french]{babel} 
\usepackage{graphicx} % Pour les images
\def\@captype{figure}
\usepackage{float}
\usepackage{multicol} % Pour faire des multi colonnes
\usepackage[export]{adjustbox} % Pour la clé 'valign'  (aligner verticalement)
\usepackage[colorlinks=true,linkcolor=black]{hyperref} % Pour qu'il y est des liens sur la table des matières
\usepackage{caption} % Utiliser plus de fonction sur caption (caption* pour ne pas afficher FIGURE-1)
\usepackage{lipsum} % Pour générer du texte pour voir comment ca rend
\usepackage{ragged2e} % Pour justifier le texte
\usepackage{ulem}
\usepackage[margin=3.2cm]{geometry}
\usepackage{forest}
\usepackage{listings}
\usepackage{xcolor}
\usepackage[section]{placeins}
\usepackage{amsmath}
\usepackage{amssymb}
\usepackage{fancybox}
\usepackage{fancyhdr}
\usepackage[none]{hyphenat}
\emergencystretch=3em
\usepackage{graphicx}
\usepackage{adjustbox}
\usepackage{float}


\usepackage{csquotes}


% Configuration de fancyhdr
\pagestyle{fancy}
\fancyhf{} % Efface tous les en-tètes et pieds de page précédents
\fancyhead[C]{Rapport de Projet} % Texte centré en haut de la page
\fancyfoot[L]{2025} % Texte en bas à  gauche du pied de page
\fancyfoot[C]{COUTANT, LITAMPHA, PETER} % Texte au centre du pied de page
\fancyfoot[R]{\thepage} % Numéro de page en bas à  droite du pied de page

\renewcommand{\headrulewidth}{0.4pt} % à‰paisseur de la ligne de séparation de l'en-tète
\renewcommand{\footrulewidth}{0.4pt} % à‰paisseur de la ligne de séparation du pied depage
\setlength{\headheight}{13.59999pt}
% Appliquer le style d'en-tète fancy aux pages de début de chapitre

\fancypagestyle{plain}{
  \fancyhf{} % Effacer tous
  \fancyhead[C]{Rapport de Projet} % Texte centré en haut de la page
  \fancyfoot[L]{L3 Informatique} % Texte en bas à  gauche du pied de page
  \fancyfoot[C]{COUTANT, LITAMPHA, PETER} % Texte au centre du pied de page
  \fancyfoot[R]{\thepage} % Numéro de page en bas à  droite du pied de page
  \renewcommand{\headrulewidth}{0.4pt} % à‰paisseur de la ligne de séparation de l'en-tète
  \renewcommand{\footrulewidth}{0.4pt} % à‰paisseur de la ligne de séparation du pied de page
}




\begin{document}

% Page de titre

% Page de titre
\begin{titlepage}
    \centering
    \vspace{3cm}
    \includegraphics[width=0.4\textwidth]{images/logo_univ.jpg}\hfill
    \includegraphics[width=0.5\textwidth]{images/logo_ufrst.png}\\
    \vspace{1.5cm}
    {\huge \textbf{Université Marie \& Louis Pasteur\\ UFR Sciences et Techniques}\par}
    \vspace{0.5cm}


    \shadowbox{\parbox{0.9\linewidth}{\centering \huge \textbf{Projet L3 : Pacman en réseau}}}\\
    \vspace{1cm}
    {\Large \textbf{Rapport de Projet} \\ Licence CMI Informatique - 3ème année \\ 2025-2026\par}
    \vspace{0.5cm}

    \includegraphics[width=0.7\textwidth]{images/entete.png}\\
    \vspace{0.5cm}

    {\LARGE \textbf{Thomas COUTANT}\\ \textbf{Benoît LITAMPHA}\\ \textbf{Auriane PETER}\par}
    \vspace{1cm}
    {\large \textbf{Encadrant : Julien BERNARD} }\\

    
\end{titlepage}


    


\chapter*{Remerciements}
\addcontentsline{toc}{chapter}{Remerciements}


\tableofcontents
\listoffigures

\chapter*{Introduction}
\addcontentsline{toc}{chapter}{Introduction}



\vfill


\chapter{Présentation du projet}

\section{Contexte académique}

Ce projet a été réalisé dans le cadre du sixième semestre de la licence informatique de l’Université Marie \& Louis Pasteur. Il s’inscrit dans un enseignement visant à mettre en pratique les connaissances acquises au cours de la formation, notamment à travers la réalisation d’un projet informatique de taille conséquente.

Le travail a été mené en groupe de trois étudiants sur une période s’étendant de la mi-octobre à la fin du mois de mars, sous l’encadrement de Julien Bernard, maître de conférences à l’Université Marie \& Louis Pasteur. Le projet porte sur le développement d’un jeu vidéo en réseau, depuis sa conception jusqu’à son implémentation.

Les objectifs pédagogiques de ce projet sont multiples. Il vise en particulier à consolider les compétences en programmation C++, à approfondir la découverte du développement de jeux vidéo, ainsi qu’à mettre en œuvre une architecture réseau de type client–serveur. Le projet permet également de développer des compétences transversales telles que le travail en équipe et la gestion d’un projet informatique sur une durée prolongée.

\section{Présentation générale du jeu \texttt{Pac-Man}}

Le jeu \texttt{Pac-Man}, apparu au début des années 1980, est un jeu d’arcade dans lequel le joueur contrôle un personnage évoluant dans un labyrinthe. L’objectif principal est de parcourir l’ensemble du labyrinthe afin de consommer toutes les \texttt{pac-gommes} qui s’y trouvent, tout en évitant d’entrer en contact avec des fantômes. Ces derniers constituent les adversaires du joueur et cherchent à bloquer ou intercepter ses déplacements. Certaines \texttt{pac-gommes} spéciales permettent temporairement à \texttt{Pac-Man} d’inverser les rôles en devenant capable d’éliminer les fantômes, ajoutant ainsi une dimension stratégique au gameplay.

\begin{figure}[H]
    \centering
    \includegraphics[width=0.8\textwidth]{images/pac_man_original.png}
    \caption{PacMan : le jeu original}
\end{figure}


Dans le cadre de ce projet, le jeu a été adapté afin d’introduire une dimension multijoueur. Une partie oppose un joueur incarnant \texttt{Pac-Man} à plusieurs joueurs incarnant des fantômes. Lorsque le nombre de joueurs humains est insuffisant, certains fantômes peuvent être contrôlés par des intelligences artificielles afin de garantir un déroulement cohérent de la partie. Cette approche permet de conserver le principe asymétrique du jeu original, tout en introduisant des interactions directes entre joueurs humains.

Le choix de \texttt{Pac-Man} comme base pour une adaptation en réseau s’explique par plusieurs raisons. Tout d’abord, ses règles simples et universellement connues permettent une prise en main rapide par les joueurs, ce qui facilite les phases de test et de démonstration. De plus, le gameplay asymétrique se prête naturellement à une séparation des rôles entre plusieurs joueurs connectés.

\section{Objectifs généraux du projet}

Les objectifs du projet peuvent être divisés en deux catégories principales : les objectifs fonctionnels, liés à l’expérience de jeu, et les objectifs techniques, liés à la conception et à l’implémentation du système.

Du point de vue fonctionnel, l’objectif principal était de concevoir un jeu Pac-Man jouable en réseau, permettant à plusieurs joueurs de participer simultanément à une même partie.

Sur le plan technique, le projet visait à mettre en \oe uvre une architecture réseau robuste reposant sur un modèle client–serveur. Le serveur devait être capable de gérer les connexions simultanées des joueurs, la création et la gestion des parties, ainsi que l’échange des données nécessaires au bon déroulement du jeu.

\chapter{Définition du sujet et problématique}

\section{Problématique du jeu en réseau}

La conception d’un jeu multijoueur en temps réel implique plusieurs contraintes techniques et architecturales. La première difficulté réside dans la latence des communications réseau, qui peut perturber la réactivité du jeu et l’expérience du joueur. Il est donc essentiel de minimiser le décalage entre les actions d’un joueur et leur effet visible pour tous les participants.

Une autre contrainte majeure concerne la synchronisation des états du jeu entre les différents clients. Chaque joueur doit avoir une vision cohérente de l’ensemble de la partie, incluant les positions de Pac-Man, des fantômes et l’état des pac-gommes. Pour garantir cette cohérence, le projet a opté pour un modèle de serveur autoritaire : toutes les décisions critiques sont validées côté serveur, qui diffuse ensuite les informations mises à jour à tous les clients. Les clients, de leur côté, envoient uniquement leurs intentions de déplacement, qui sont ensuite traitées et synchronisées par le serveur. Les données transitent via des buffers et des queues avant d’être appliquées, assurant un traitement ordonné et fiable des événements du jeu.

\section{Choix techniques}

Le projet a été développé en C++ en utilisant la bibliothèque \textit{Gamedev Framework} (gf), adaptée au développement de jeux vidéo et compatible avec la gestion de l’architecture réseau. La communication réseau repose exclusivement sur le protocole TCP, assurant la fiabilité des échanges entre le serveur et les clients. Les messages échangés sont sérialisés afin de structurer et standardiser les informations transmises.

Le serveur a été conçu selon un modèle multithreadé, reflétant la logique de fonctionnement du jeu : un thread principal gère le serveur et le lobby, tandis que d’autres threads s’occupent des rooms, de la boucle de jeu, des fantômes contrôlés par l’IA et de la communication avec les clients. Cette architecture permet de traiter simultanément les interactions multiples des joueurs et d’assurer une exécution fluide et cohérente des parties.

\section{Répartition des rôles}

La réalisation de ce projet a été organisée de manière collaborative entre les trois membres du groupe. 

Thomas Coutant a principalement travaillé sur le développement du serveur, ainsi que sur la conception de la logique du jeu. Il a également développé un prototype minimal de démarrage, permettant de tester la communication client–serveur avec un simple cube représentant un joueur en mouvement.

Auriane Peter s’est concentrée sur le développement du client, en implémentant l’interface et la gestion des interactions côté joueur.

Benoît Litampha a assuré la communication entre le serveur et le client, participant également à certaines tâches sur le serveur et le client lorsque nécessaire. Cette répartition a permis une progression simultanée sur plusieurs fronts du projet, tout en garantissant une cohérence globale de l’application.

%\chapter{L'Organisation du stage}


\chapter{Architecture et conception}

\section{Architecture générale}

Description de l’architecture client-serveur, diagrammes éventuels.

\section{Structure du serveur}

Gestion des joueurs, des parties, des salles (lobby),
synchronisation de l’état du jeu.

\section{Structure du client}

Affichage, gestion des entrées utilisateur,
réception et traitement des messages réseau.

\section{Protocoles et échanges réseau}

Description des trames, types de messages,
gestion des événements (connexion, mouvement, fin de partie).


%\section{L'application}

%\section{La gestion des tickets}

\chapter{Réalisation et déroulement du projet}

\section{Mise en place de l’environnement de développement}

Outils utilisés, configuration, compilation.

\section{Implémentation des fonctionnalités principales}

\subsection*{Gestion du déplacement et des collisions}

\subsection*{Synchronisation des joueurs}

\subsection*{Gestion des fantômes et de l’IA}

\subsection*{Système de score et conditions de victoire}

\section{Gestion du multijoueur}

Connexion des clients, création des parties,
gestion des déconnexions.

\section{Difficultés rencontrées}

Problèmes techniques, bugs réseau,
choix remis en question.



\chapter{Bilan et perspectives}

\section{Compétences acquises}

Programmation réseau, architecture logicielle,
travail de conception.

\section{Résultat final}

Fonctionnalités implémentées,
objectifs atteints ou partiellement atteints.

\section{Améliorations possibles}

idee.txt



% Page de conclusion 
\chapter*{Conclusion}
\addcontentsline{toc}{chapter}{Conclusion}

Ce projet a permis de mettre en pratique les connaissances acquises
durant la formation en informatique, en particulier dans le domaine
du développement réseau et de la conception logicielle.




% Page de conclusion 
\chapter*{Sitographie}
\addcontentsline{toc}{chapter}{Sitographie} % Ajout de la sitographie sur la table des matières sans ètre numéroté


% Page de résumé
\newpage
\begin{center}
    \vspace*{\fill} % Espace vertical
    \section*{Résumé}
    \addcontentsline
    {toc}{chapter}{Résumé}
    \begin{justify}

    \end{justify}
    \vspace*{\fill} % Espace vertical

    \section*{Mots-clés}
    \addcontentsline
    {toc}{chapter}{Mots-Clés}
    \begin{center}
     C++ - PacMan - GF - Réseau - [AUTRE]
    \end{center}
    \vspace*{\fill} % Espace vertical

    \section*{Abstract}
    \addcontentsline
    {toc}{chapter}{Abstract}
    \begin{justify}

    \end{justify}
    \vspace*{\fill} % Espace vertical

    \section*{Key-Words}
    \addcontentsline
    {toc}{chapter}{Key-Words}
    \begin{center}
    MathLive - JavaScript - TypeScript - HTML - Drag \& Drop
    \end{center}
    \vspace*{\fill} % Espace vertical
    
    
\end{center}

\end{document}