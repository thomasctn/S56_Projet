\documentclass[a4paper, 11pt]{report}
\usepackage[utf8]{inputenc}   % Pour lire les fichiers UTF-8
\usepackage[T1]{fontenc}      % Pour une meilleure gestion des caractères accentués
\usepackage{textcomp}         % Pour quelques symboles supplémentaires
\usepackage{lmodern}          % Police compatible avec T1
\usepackage[english,french]{babel} 
\usepackage{graphicx} % Pour les images
\def\@captype{figure}
\usepackage{float}
\usepackage{multicol} % Pour faire des multi colonnes
\usepackage[export]{adjustbox} % Pour la clé 'valign'  (aligner verticalement)
\usepackage[colorlinks=true,linkcolor=black]{hyperref} % Pour qu'il y est des liens sur la table des matières
\usepackage{caption} % Utiliser plus de fonction sur caption (caption* pour ne pas afficher FIGURE-1)
\usepackage{lipsum} % Pour générer du texte pour voir comment ca rend
\usepackage{ragged2e} % Pour justifier le texte
\usepackage{ulem}
\usepackage[margin=3.2cm]{geometry}
\usepackage{forest}
\usepackage{listings}
\usepackage{xcolor}
\usepackage[section]{placeins}
\usepackage{amsmath}
\usepackage{amssymb}
\usepackage{fancybox}
\usepackage{fancyhdr}
\usepackage[none]{hyphenat}
\emergencystretch=3em

\usepackage[backend=biber,style=numeric]{biblatex}
\addbibresource{sitographie.bib}
\usepackage{csquotes}


% Configuration de fancyhdr
\pagestyle{fancy}
\fancyhf{} % Efface tous les en-tètes et pieds de page précédents
\fancyhead[C]{Rapport de Stage} % Texte centré en haut de la page
\fancyfoot[L]{2025} % Texte en bas à  gauche du pied de page
\fancyfoot[C]{Thomas COUTANT} % Texte au centre du pied de page
\fancyfoot[R]{\thepage} % Numéro de page en bas à  droite du pied de page

\renewcommand{\headrulewidth}{0.4pt} % à‰paisseur de la ligne de séparation de l'en-tète
\renewcommand{\footrulewidth}{0.4pt} % à‰paisseur de la ligne de séparation du pied depage
\setlength{\headheight}{13.59999pt}
% Appliquer le style d'en-tète fancy aux pages de début de chapitre

\fancypagestyle{plain}{
  \fancyhf{} % Effacer tous
  \fancyhead[C]{Rapport de Stage} % Texte centré en haut de la page
  \fancyfoot[L]{L2 Informatique} % Texte en bas à  gauche du pied de page
  \fancyfoot[C]{Thomas COUTANT} % Texte au centre du pied de page
  \fancyfoot[R]{\thepage} % Numéro de page en bas à  droite du pied de page
  \renewcommand{\headrulewidth}{0.4pt} % à‰paisseur de la ligne de séparation de l'en-tète
  \renewcommand{\footrulewidth}{0.4pt} % à‰paisseur de la ligne de séparation du pied de page
}




\begin{document}

% Page de titre

% Page de titre
\begin{titlepage}
    \centering
    \vspace{3cm}
    \includegraphics[width=0.4\textwidth]{images/logo_univ.jpg}\hfill
    \includegraphics[width=0.5\textwidth]{images/logo_ufrst.png}\\
    \vspace{1.5cm}
    {\huge \textbf{Université Marie \& Louis Pasteur\\ UFR Sciences et Techniques}\par}
    \vspace{0.5cm}


    \shadowbox{\parbox{0.9\linewidth}{\centering \huge \textbf{Projet L3 : Pacman en réseau}}}\\
    \vspace{1cm}
    {\Large \textbf{Rapport de Projet} \\ Licence CMI Informatique - 3ème année \\ 2025-2026\par}
    \vspace{0.5cm}

    \includegraphics[width=1\textwidth]{images/entete.png}\\
    \vspace{0.5cm}

    {\LARGE \textbf{Thomas COUTANT}\\ \textbf{Benoît LITAMPHA}\\ \textbf{Auriane PETER}\par}
    \vspace{1cm}
    {\large \textbf{Encadrant : Julien BERNARD} }\\

    
\end{titlepage}


    


\chapter*{Remerciements}
\addcontentsline{toc}{chapter}{Remerciements}


\tableofcontents
\listoffigures

\chapter*{Introduction} % * pour ne pas avoir de numéro de chapitre
\addcontentsline{toc}{chapter}{Introduction}


\vfill

\chapter{La structure d’accueil de mon stage : EBOP}


\section{Présentation de l'organisation}




\section{L’environnement de travail}


\vspace{0.5cm}


\chapter{Définition du sujet de stage}
\section{Contexte et problématique}


\section{Objectifs du stage}




\section{Collaboration et intégration}


%\chapter{L'Organisation du stage}


\section{Équipement et outils}


%\section{L'application}

%\section{La gestion des tickets}

\chapter{Contenu et déroulement du stage}

\section{Exploration des outils et premières implémentations}


\subsection*{12–13 juin : premiers essais d’interaction}


\subsection*{16 juin : conception d’un déclencheur de transformation}



\subsection*{17 juin : implémentation du système de bloc par clic long}


\subsection*{Ajouts fonctionnels : zoom, lecture vocale, duplication}







\subsection*{Conception du tableau de contrôle}



\section{Intégration, adaptation et validation croisée}

\subsection*{Collaboration avec un autre stagiaire}


\section{Consolidation, optimisation et finalisation du prototype}



\chapter{Bilan}

\section{Acquis techniques}




\section{Résultat final}


% Page de conclusion 
\chapter*{Conclusion}
\addcontentsline{toc}{chapter}{Conclusion}



% Page de conclusion 
\chapter*{Sitographie}
\addcontentsline{toc}{chapter}{Sitographie} % Ajout de la sitographie sur la table des matières sans ètre numéroté

\printbibliography[heading=none]


% Page de résumé
\newpage
\begin{center}
    \vspace*{\fill} % Espace vertical
    \section*{Résumé}
    \addcontentsline
    {toc}{chapter}{Résumé}
    \begin{justify}

    \end{justify}
    \vspace*{\fill} % Espace vertical

    \section*{Mots-clés}
    \addcontentsline
    {toc}{chapter}{Mots-Clés}
    \begin{center}
     MathLive - JavaScript - TypeScript - HTML - Drag \& Drop
    \end{center}
    \vspace*{\fill} % Espace vertical

    \section*{Abstract}
    \addcontentsline
    {toc}{chapter}{Abstract}
    \begin{justify}

    \end{justify}
    \vspace*{\fill} % Espace vertical

    \section*{Key-Words}
    \addcontentsline
    {toc}{chapter}{Key-Words}
    \begin{center}
    MathLive - JavaScript - TypeScript - HTML - Drag \& Drop
    \end{center}
    \vspace*{\fill} % Espace vertical
    
    
\end{center}

\end{document}