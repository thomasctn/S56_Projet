\documentclass[a4paper, 11pt]{report}
\usepackage[utf8]{inputenc}   % Pour lire les fichiers UTF-8
\usepackage[T1]{fontenc}      % Pour une meilleure gestion des caractères accentués
\usepackage{textcomp}         % Pour quelques symboles supplémentaires
\usepackage{lmodern}          % Police compatible avec T1
\usepackage[english,french]{babel} 
\usepackage{graphicx} % Pour les images
\def\@captype{figure}
\usepackage{float}
\usepackage{multicol} % Pour faire des multi colonnes
\usepackage[export]{adjustbox} % Pour la clé 'valign'  (aligner verticalement)
\usepackage[colorlinks=true,linkcolor=black]{hyperref} % Pour qu'il y est des liens sur la table des matières
\usepackage{caption} % Utiliser plus de fonction sur caption (caption* pour ne pas afficher FIGURE-1)
\usepackage{lipsum} % Pour générer du texte pour voir comment ca rend
\usepackage{ragged2e} % Pour justifier le texte
\usepackage{ulem}
\usepackage[margin=3.2cm]{geometry}
\usepackage{forest}
\usepackage{listings}
\usepackage{xcolor}
\usepackage[section]{placeins}
\usepackage{amsmath}
\usepackage{amssymb}
\usepackage{fancybox}
\usepackage{fancyhdr}
\usepackage[none]{hyphenat}
\emergencystretch=3em
\usepackage{graphicx}
\usepackage{adjustbox}
\usepackage{float}


\usepackage{csquotes}


% Configuration de fancyhdr
\pagestyle{fancy}
\fancyhf{} % Efface tous les en-tètes et pieds de page précédents
\fancyhead[C]{Rapport de Projet} % Texte centré en haut de la page
\fancyfoot[L]{2025} % Texte en bas à  gauche du pied de page
\fancyfoot[C]{COUTANT, LITAMPHA, PETER} % Texte au centre du pied de page
\fancyfoot[R]{\thepage} % Numéro de page en bas à  droite du pied de page

\renewcommand{\headrulewidth}{0.4pt} % à‰paisseur de la ligne de séparation de l'en-tète
\renewcommand{\footrulewidth}{0.4pt} % à‰paisseur de la ligne de séparation du pied depage
\setlength{\headheight}{13.59999pt}
% Appliquer le style d'en-tète fancy aux pages de début de chapitre

\fancypagestyle{plain}{
  \fancyhf{} % Effacer tous
  \fancyhead[C]{Rapport de Projet} % Texte centré en haut de la page
  \fancyfoot[L]{L3 Informatique} % Texte en bas à  gauche du pied de page
  \fancyfoot[C]{COUTANT, LITAMPHA, PETER} % Texte au centre du pied de page
  \fancyfoot[R]{\thepage} % Numéro de page en bas à  droite du pied de page
  \renewcommand{\headrulewidth}{0.4pt} % à‰paisseur de la ligne de séparation de l'en-tète
  \renewcommand{\footrulewidth}{0.4pt} % à‰paisseur de la ligne de séparation du pied de page
}




\begin{document}

% Page de titre

% Page de titre
\begin{titlepage}
    \centering
    \vspace{3cm}
    \includegraphics[width=0.4\textwidth]{images/logo_univ.jpg}\hfill
    \includegraphics[width=0.5\textwidth]{images/logo_ufrst.png}\\
    \vspace{1.5cm}
    {\huge \textbf{Université Marie \& Louis Pasteur\\ UFR Sciences et Techniques}\par}
    \vspace{0.5cm}


    \shadowbox{\parbox{0.9\linewidth}{\centering \huge \textbf{Projet L3 : Pacman en réseau}}}\\
    \vspace{1cm}
    {\Large \textbf{Rapport de Projet} \\ Licence CMI Informatique - 3ème année \\ 2025-2026\par}
    \vspace{0.5cm}

    \includegraphics[width=1\textwidth]{images/entete.png}\\
    \vspace{0.5cm}

    {\LARGE \textbf{Thomas COUTANT}\\ \textbf{Benoît LITAMPHA}\\ \textbf{Auriane PETER}\par}
    \vspace{1cm}
    {\large \textbf{Encadrant : Julien BERNARD} }\\

    
\end{titlepage}


    


\chapter*{Remerciements}
\addcontentsline{toc}{chapter}{Remerciements}


\tableofcontents
\listoffigures

\chapter*{Introduction}
\addcontentsline{toc}{chapter}{Introduction}

Ce projet s’inscrit dans le cadre de la licence 3 Informatique.
L’objectif est la conception et la réalisation d’un jeu de type Pac-Man
multijoueur en réseau, mettant en œuvre des notions de programmation,
d’architecture logicielle et de communication client-serveur.

\vfill


\chapter{Présentation du projet}

\section{Contexte académique}

Présentation du cadre universitaire du projet (UE, semestre, travail en groupe ou individuel,
objectifs pédagogiques).

\section{Présentation générale du jeu Pac-Man}

Rappel du principe du jeu Pac-Man original et justification de son choix
pour une adaptation en réseau.

\section{Objectifs généraux du projet}

Objectifs fonctionnels (jeu jouable, multijoueur, synchronisation)
et objectifs techniques (réseau, performance, architecture).


\chapter{Définition du sujet et problématique}

\section{Problématique du jeu en réseau}

Contraintes liées au multijoueur en temps réel :
latence, synchronisation des états, cohérence entre clients.

\section{Choix techniques}

Choix du langage, des bibliothèques, du modèle réseau
(client-serveur, protocole, gestion des messages).

\section{Répartition des rôles}

Description de la collaboration, répartition des tâches et organisation du travail.


%\chapter{L'Organisation du stage}


\chapter{Architecture et conception}

\section{Architecture générale}

Description de l’architecture client-serveur, diagrammes éventuels.

\section{Structure du serveur}

Gestion des joueurs, des parties, des salles (lobby),
synchronisation de l’état du jeu.

\section{Structure du client}

Affichage, gestion des entrées utilisateur,
réception et traitement des messages réseau.

\section{Protocoles et échanges réseau}

Description des trames, types de messages,
gestion des événements (connexion, mouvement, fin de partie).


%\section{L'application}

%\section{La gestion des tickets}

\chapter{Réalisation et déroulement du projet}

\section{Mise en place de l’environnement de développement}

Outils utilisés, configuration, compilation.

\section{Implémentation des fonctionnalités principales}

\subsection*{Gestion du déplacement et des collisions}

\subsection*{Synchronisation des joueurs}

\subsection*{Gestion des fantômes et de l’IA}

\subsection*{Système de score et conditions de victoire}

\section{Gestion du multijoueur}

Connexion des clients, création des parties,
gestion des déconnexions.

\section{Difficultés rencontrées}

Problèmes techniques, bugs réseau,
choix remis en question.



\chapter{Bilan et perspectives}

\section{Compétences acquises}

Programmation réseau, architecture logicielle,
travail de conception.

\section{Résultat final}

Fonctionnalités implémentées,
objectifs atteints ou partiellement atteints.

\section{Améliorations possibles}

idee.txt



% Page de conclusion 
\chapter*{Conclusion}
\addcontentsline{toc}{chapter}{Conclusion}

Ce projet a permis de mettre en pratique les connaissances acquises
durant la formation en informatique, en particulier dans le domaine
du développement réseau et de la conception logicielle.




% Page de conclusion 
\chapter*{Sitographie}
\addcontentsline{toc}{chapter}{Sitographie} % Ajout de la sitographie sur la table des matières sans ètre numéroté


% Page de résumé
\newpage
\begin{center}
    \vspace*{\fill} % Espace vertical
    \section*{Résumé}
    \addcontentsline
    {toc}{chapter}{Résumé}
    \begin{justify}

    \end{justify}
    \vspace*{\fill} % Espace vertical

    \section*{Mots-clés}
    \addcontentsline
    {toc}{chapter}{Mots-Clés}
    \begin{center}
     MathLive - JavaScript - TypeScript - HTML - Drag \& Drop
    \end{center}
    \vspace*{\fill} % Espace vertical

    \section*{Abstract}
    \addcontentsline
    {toc}{chapter}{Abstract}
    \begin{justify}

    \end{justify}
    \vspace*{\fill} % Espace vertical

    \section*{Key-Words}
    \addcontentsline
    {toc}{chapter}{Key-Words}
    \begin{center}
    MathLive - JavaScript - TypeScript - HTML - Drag \& Drop
    \end{center}
    \vspace*{\fill} % Espace vertical
    
    
\end{center}

\end{document}